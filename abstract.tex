\begin{foreignabstract}
\nohyphens{
Since 1950, the software and hardware technology for robotic systems is
developed, and three paradigms were idealized to try to answer the question: what
is the correct way to build an intelligent autonomous system? The comprehension
of the various robotic architectures and robotic cases, and their advantages and
disadvantages allow analysis, and improvement proposals. The study of robotic
architectures and how they are used allow the understanding of how different
components and tools associated with a paradigm can build an artificial intelligence
robot. The objective of this work is the development of an autonomous robotic
system in an hybrid three-layers architecture, i.e., the implementation of the
Executive and Planner layers. The case study for this thesis DORIS, a mobile robot
for remote supervision, diagnosis, and data acquisition on offshore facilities. Tests
and results showed the performance of the proposed architecture and a comparative
analysis highlighted the modularity, flexibility and robustness of the solution.
}

\end{foreignabstract}

