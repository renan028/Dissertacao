\begin{foreignabstract}
\nohyphens{
Intelligent autonomous systems research grows rapidly
in academia and industry, as they are a multi-disciplinary challenge,
financially attractive, robust with high reliability, they improve
human living conditions and labor, and minimize possible human errors in
high-risk systems. Since 1950, the software and hardware technology for these
systems is developed, and three paradigms were idealized to try to answer the
question: what is the correct way to build an intelligent autonomous system? The
comprehension of the various robotic architectures and robotic cases, and their
advantages and disadvantages allow analysis, and improvement proposals. The
study of robotic architectures and how they are used allow the
understanding of how different components and tools associated with a paradigm
can build an artificial intelligence robot. The objective of this work is
the implementation of a robotic architecture for mobile inspection robots and
similar applications. The case study for this thesis DORIS, a mobile robot for
remote supervision, diagnosis, and data acquisition on offshore facilities.
Therefore, an autonomous system is developed for DORIS, following the
methodology and criteria evaluation in accordance with the state of the art.
Tests and results showed the positive performance of the proposed architecture
and a comparative analysis highlighted the intuitive, modular and
flexible implementation of the solution.
}

\end{foreignabstract}

