\begin{abstract}
\nohyphens{
A tecnologia de software e hardware para constru��o de sistemas rob�ticos �
desenvolvida desde 1950, e at� hoje alguns paradigmas j� foram idealizados para
responder a pergunta: qual a maneira correta de construir um sistema aut�nomo?
A compreens�o e a avalia��o de casos das diversas arquiteturas rob�ticas
permitem a an�lise cr�tica dos sistemas, e propostas de melhoramentos. O estudo
de arquiteturas rob�ticas e como suas inst�ncias s�o usadas para uma
determinada aplica��o possibilita o aprendizado das diferentes formas que os
componentes e ferramentas associados a um paradigma podem ser usados para a
constru��o de uma intelig�ncia artificial rob�tica. O objetivo deste trabalho
� o desenvolvimento de um sistema rob�tico aut�nomo em uma arquitetura
 h�brida de tr�s camadas, isto �, a implementa��o das camadas Planejador e
 Executivo. O estudo de caso para esta disserta��o � o sistema rob�tico DORIS,
 rob� \textit{offshore} para inspe��o em plataformas de petr�leo. Resultados
 e testes mostram o desempenho da arquitetura e uma an�lise comparativa
 evidencia a modularidade, flexibilidade, e robustez da solu��o.
}

\end{abstract}

